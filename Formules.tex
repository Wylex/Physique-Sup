%Préamble {{{1
\documentclass[fleqn]{article}

\usepackage{amssymb}
\usepackage{amsmath}
\usepackage{amsthm}
\usepackage{verbatim}
\usepackage{booktabs}
\usepackage{mathrsfs}

\theoremstyle{definition} \newtheorem*{defi}{D\'efinition}
\theoremstyle{definition} \newtheorem*{theo}{Th\'eor\`eme}
\theoremstyle{definition} \newtheorem*{coro}{Corollaire}
\theoremstyle{remark} \newtheorem*{rqs}{Remarques}
\theoremstyle{definition} \newtheorem*{prop}{Propri\'et\'e}
\newcommand{\ra}[1]{\renewcommand{\arraystretch}{#1}}
\ra{1.3}

\title{Formules}
\date{}

\begin{document}
\maketitle

%Formules {{{1

\newcounter{ct}
\stepcounter{ct}

\section{Signaux}
\subsection{Oscillateur harmonique $^\thect$}
\begin{enumerate}
	\item $\vec{F}_{elas} = -k(l - l_0)\vec{u}_{ext}$
	\item $\frac{d^2u}{dt^2} + \omega_0^2 u = 0$ \\
		$u(t) = X_m \cos(\omega_0 t + \varphi)$\\
		$u(t) = A\cos(\omega_0 t) + B\sin(\omega_0 t)$
	\item  $E_{pe} = \frac{1}{2}k(l - l_0)^2\ (+cte)$
\end{enumerate}
\stepcounter{ct}

\subsection{Propagation signal $^\thect$}
\begin{enumerate}
	\item $s(x,t) = f(x - ct)$ (resp. $f(x + ct)$) si onde se propage $x$ croissants (resp. d\'ecroissants)
\end{enumerate}
\stepcounter{ct}

\subsection{Onde progressive sinuso\"idale $^\thect$}
\begin{enumerate}
	\item $s(x,t) = A\cos(\omega t - kx + \varphi)$
	\item $k =  \frac{\omega}{c}$
	\item $\varphi_2 - \varphi_1 = -\frac{2\pi}{T} (t_2 - t_1)$
\end{enumerate}
\stepcounter{ct}

\subsection{Superposition d'ondes progressives sinuso\"idales $^\thect$}
\begin{enumerate}
	\item $s(M,t) = A_1\cos(\omega t + \varphi_1 (M)) + A_2\cos(\omega t + \varphi_2 (M)) = A_r\cos(\omega t + \varphi_r (M))$
	\item $s(x,t) = C\cos(kx + \psi)\cos(\omega t + \varphi)$
	\item $L = n \frac{\lambda}{2}$
\end{enumerate}
\stepcounter{ct}

\section{Ondes lumineuses}
\subsection{Mod\`ele g\'eom\'etrique de la lumi\`ere $^\thect$}
\begin{enumerate}
	\item $n = \frac{c}{v}$
	\item $n_1 \sin(i_1) = n_2 \sin(i_2)$
\end{enumerate}
\stepcounter{ct}

\subsection{Formation des images $^\thect$}
\stepcounter{ct}

\subsection{Lentilles minces sph\'eriques $^\thect$}
\begin{enumerate}
	\item $\gamma = \frac{\overline{F'A'}}{-f'} = \frac{f'}{\overline{FA}}$
	\item $\overline{F'A'}.\overline{FA} = \overline{F'O}.\overline{FO} = -f'^2$
	\item $\gamma = \frac{\overline{OA'}}{\overline{OA}}$
	\item $\frac{1}{\overline{OA'}} - \frac{1}{\overline{OA}} = \frac{1}{f'}$
\end{enumerate}
\stepcounter{ct}

\section{Structure mati\`ere}
\subsection{Introduction au monde quantique $^{\thect}$}
\begin{enumerate}
	\item $E = $
	\item $\vec{p} = $
	\item $\lambda_{DB} = $
\end{enumerate}
\stepcounter{ct}

\subsection{Quantification d'\'energie dans l'atome d'hydrog\`ene $^{\thect}$}
\begin{enumerate}
	\item $E_n = \frac{13,6}{n}$?
\end{enumerate}
\stepcounter{ct}

\subsection{Structure \'electronique des atomes $^{\thect}$}
\begin{enumerate}
	\item $s < p < f < d < \hdots $
\end{enumerate}
\stepcounter{ct}

\subsection{Classification p\'eriodique des \'el\'ements $^{\thect}$}
\begin{enumerate}
	\item a
\end{enumerate}
\stepcounter{ct}

\subsection{Structure des mol\'ecules $^{\thect}$}
\begin{enumerate}
	\item a
\end{enumerate}
\stepcounter{ct}


\section{\'Electricit\'e}
\subsection{Lois g\'en\'erales de l'\'electrocin\'etique $^{\thect}$}
\begin{enumerate}
	\item $i = \frac{dq}{dt}$
	\item $u_{AB} = V_A - V_B$
	\item $p = u.i$ (puissance re\c{c}ue pour dipole en convention r\'ecepteur)
\end{enumerate}
\stepcounter{ct}

\subsection{Mod\'elisation des dip\^oles usuels $^{\thect}$}
\begin{enumerate}
	\item Conducteur ohmique: convention r\'ecepteur $u = Ri$
	\item Condensateur: convention r\'ecepteur $i = C\frac{du}{dt}$
	\item Bobine: convention r\'ecepteur $u = L\frac{di}{dt}$
	\item $E_{cond}(t) = \frac{1}{2} Cu^2(t)$
	\item $E_{bob}(t) = \frac{1}{2} Li^2(t)$
	\item Charge condensateur $Q = Cu$
\end{enumerate}
\stepcounter{ct}

\subsection{R\'egime permanent $^{\thect}$}
\begin{enumerate}
	\item M\'ethode pour trouver des inconnues dans un circuit en r\'egime permanent
\end{enumerate}
\stepcounter{ct}

\subsection{R\'egimes transitoires des circuits lin\'eaires du premier ordre $^{\thect}$}
\begin{enumerate}
	\item M\'ethode d'\'etude d'un r\'egime transitoire de premier ordre
	\item $\frac{du}{dt} + \frac{u}{\tau} = \frac{E}{\tau}:$ On calcule $A$ avec les conditions initiales \\
		$u(t) = Ae^{-t/\tau} + E$
	\item Tension bornes condensateur, intensit\'e bornes bobine continues
\end{enumerate}
\stepcounter{ct}

\subsection{R\'egimes transitoires des syst\`emes lin\'eaires du deuxi\`eme ordre. Oscillateurs amortis $^{\thect}$}
\begin{enumerate}
	\item $\frac{d^2u}{dt^2} + \frac{\omega_0}{Q} \frac{du}{dt} + \omega_0^2 u = \omega_0^2 E$\\
		$\frac{d^2u}{dt^2} + \frac{1}{\tau} \frac{du}{dt} + \omega_0^2 u = \omega_0^2 E$\\
		R\'esolution de l'\'equation homog\`ene
		\begin{enumerate}
			\item $Q < \frac{1}{2}$, $u(t) = Ae^{r_1 t} + B e^{r_2 t}$
			\item $Q = \frac{1}{2}$, $u(t) = (At + B)e^{-\omega_0 t}$
			\item $Q > \frac{1}{2}$: soit $\Omega = \omega_0 \sqrt{1 - \frac{1}{4Q^2}}$,
				\\ $u(t) = e^{\frac{-\omega_0 t}{2Q}}(A\cos(\Omega t) + B\sin(\Omega t))$\\
				$u(t) = C e^{\frac{-\omega_0 t}{2Q}} \cos(\Omega t + \varphi)$
		\end{enumerate}
\end{enumerate}
\stepcounter{ct}

\section{Syst\`emes physico-chimique}
\subsection{Description et \'evolution d'un syst\`eme vers l'\'equilibre chimique $^{\thect}$}
\begin{enumerate}
	\item S-G: sublimation, condensation\\
		G-L: liquefaction, Vaporisation\\
		L-S: Solidification, Fusion
	\item Types transformation mati\`ere: physique (ch \'etats), chimique, nucl\'eaire
	\item Titre molaire: $x_i = \frac{n_i}{\sum n_k}$
	\item Pression partielle (gp): $P_i V = n_i RT \quad P = \sum P_i$
	\item $Q(t) = \frac{\prod a(produits)}{\prod a(reac)} \rightarrow K^0$. Utilit\'es:
		\begin{enumerate}
			\item Prevoire sens de r\'eaction
			\item D\'eterminer sens de r\'eaction
				\begin{enumerate}
					\item $K^0 \ll 1$ alors $x_{eq}$ n\'egligeable devant concentration initiales
					\item $K^0 \gg 1$ alors $x_{eq} \approx x_{max} \Rightarrow x_{max} - x_{eq} = \epsilon$ avec $\epsilon \ll x_{max}$
				\end{enumerate}
		\end{enumerate}
\end{enumerate}
\stepcounter{ct}

\subsection{Cin\'etique chimique $^{\thect}$}
\begin{enumerate}
	\item Vitesse globale: $v(t) = \frac{1}{V} \frac{d \xi}{dt} = $ (\`a $V$ cte) $\frac{1}{\nu_i} \frac{d[A_i]}{dt}$ \\
		Vitesse formation r\'eactant: $v(A_i) = \frac{1}{V} \frac{d n(A_i)}{dt} = $ (\`a $V$ cte) $\frac{d[A_i]}{dt} > 0$
	\item Influence concentration: $v = k[R_1]^{\beta_1} [R_2]^{\beta_2} \hdots [R_N]^{\beta_N}$\\
		Influence temp\'erature: $k = A \exp (-\frac{E_A}{RT})$
	\item D\'etermination des ordres de r\'eaction: $\alpha A \rightarrow$ produits
	\begin{enumerate}
		\item M\'ethode int\'egrale: si on connait \'evolution de la concentration du r\'eactif
		\item Temps de demi-r\'eaction: si on connait \'evolution du temps de demi-reac en fonction concentration initiale
		\item M\'ethode diff\'erentielle: si on n'a pas d'id\'ee sur l'ordre de grandeur
	\end{enumerate}
	\item R\'eactions du type $\alpha A + \beta B \rightarrow $ produits
	\begin{enumerate}
		\item M\'elange stoechiom\'etrique pour d\'eterminer $p+q$
		\item D\'eg\'en\'erescence de l'ordre pour d\'eterminer $p$ et $q$
	\end{enumerate}
\end{enumerate}
\stepcounter{ct}



\end{document}
