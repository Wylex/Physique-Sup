%Préamble {{{1
\documentclass[fleqn]{article}

\usepackage{amssymb}
\usepackage{amsmath}
\usepackage{amsthm}
\usepackage{verbatim}
\usepackage{booktabs}
\usepackage{mathrsfs}

\theoremstyle{definition} \newtheorem*{defi}{D\'efinition}
\theoremstyle{definition} \newtheorem*{theo}{Th\'eor\`eme}
\theoremstyle{definition} \newtheorem*{coro}{Corollaire}
\theoremstyle{remark} \newtheorem*{rqs}{Remarques}
\theoremstyle{definition} \newtheorem*{prop}{Propri\'et\'e}
\newcommand{\ra}[1]{\renewcommand{\arraystretch}{#1}}
\ra{1.3}

\title{Formules}
\date{}

\begin{document}
\maketitle

%Formules {{{1

\newcounter{ct}
\stepcounter{ct}

\section{Signaux}
\subsection{Oscillateur harmonique $^\thect$}
\begin{enumerate}
	\item $\vec{F}_{elas} = -k(l - l_0)\vec{u}_{ext}$
	\item $\frac{d^2u}{dt^2} + \omega_0^2 u = 0$ \\
		$u(t) = X_m \cos(\omega_0 t + \varphi)$\\
		$u(t) = A\cos(\omega_0 t) + B\sin(\omega_0 t)$
	\item  $E_{pe} = \frac{1}{2}k(l - l_0)^2\ (+cte)$
\end{enumerate}
\stepcounter{ct}

\subsection{Propagation signal $^\thect$}
\begin{enumerate}
	\item $s(x,t) = f(x - ct)$ (resp. $f(x + ct)$) si onde se propage $x$ croissants (resp. d\'ecroissants)
\end{enumerate}
\stepcounter{ct}

\subsection{Onde progressive sinuso\"idale $^\thect$}
\begin{enumerate}
	\item $s(x,t) = A\cos(\omega t - kx + \varphi)$
	\item $k =  \frac{\omega}{c}$
	\item $\varphi_2 - \varphi_1 = -\frac{2\pi}{T} (t_2 - t_1)$
\end{enumerate}
\stepcounter{ct}

\subsection{Superposition d'ondes progressives sinuso\"idales $^\thect$}
\begin{enumerate}
	\item $s(M,t) = A_1\cos(\omega t + \varphi_1 (M)) + A_2\cos(\omega t + \varphi_2 (M)) = A_r\cos(\omega t + \varphi_r (M))$
	\item $s(x,t) = C\cos(kx + \psi)\cos(\omega t + \varphi)$
	\item $L = n \frac{\lambda}{2}$
\end{enumerate}
\stepcounter{ct}

\section{Ondes lumineuses}
\subsection{Mod\`ele g\'eom\'etrique de la lumi\`ere $^\thect$}
\begin{enumerate}
	\item $n = \frac{c}{v}$
	\item $n_1 \sin(i_1) = n_2 \sin(i_2)$
\end{enumerate}
\stepcounter{ct}

\subsection{Formation des images $^\thect$}
\stepcounter{ct}

\subsection{Lentilles minces sph\'eriques $^\thect$}
\begin{enumerate}
	\item $\gamma = \frac{\overline{F'A'}}{-f'} = \frac{f'}{\overline{FA}}$
	\item $\overline{F'A'}.\overline{FA} = \overline{F'O}.\overline{FO} = -f'^2$
	\item $\gamma = \frac{\overline{OA'}}{\overline{OA}}$
	\item $\frac{1}{\overline{OA'}} - \frac{1}{\overline{OA}} = \frac{1}{f'}$
\end{enumerate}
\stepcounter{ct}

\section{\'Electricit\'e}
\subsection{Lois g\'en\'erales de l'\'electrocin\'etique $^\thect$}
\begin{enumerate}
	\item $i = \frac{dq}{dt}$
	\item $u_{AB} = V_A - V_B$
	\item $p = u.i$ (puissance re\c{c}ue pour dipole en convention r\'ecepteur)
\end{enumerate}
\stepcounter{ct}

\subsection{Mod\'elisation des dip\^oles usuels $^\thect$}
\begin{enumerate}
	\item Conducteur ohmique: convention r\'ecepteur $u = Ri$
	\item Condensateur: convention r\'ecepteur $i = C\frac{du}{dt}$
	\item Bobine: convention r\'ecepteur $u = L\frac{di}{dt}$
	\item $E_{cond}(t) = \frac{1}{2} Cu^2(t)$
	\item $E_{bob}(t) = \frac{1}{2} Li^2(t)$
\end{enumerate}
\stepcounter{ct}

\subsection{R\'egime permanent $^{\thect}$}
\begin{enumerate}
	\item M\'ethode pour trouver des inconnues dans un circuit en r\'egime permanent
\end{enumerate}
\stepcounter{ct}


\end{document}
